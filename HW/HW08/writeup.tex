\documentclass[12pt]{article}
\usepackage{fullpage,enumitem,amsmath,amssymb,graphicx}

\begin{document}

  \begin{center}
  {\Large CSED342 Spring 2021 Homework 8 \vspace{10pt}}

    \begin{tabular}{rl}
      Student ID: & 20190084 \\
      Name: & Minjae Gwon \\
    \end{tabular}
  \end{center}

  By turning in this assignment, I agree by the POSTECH honor code and declare that all of this is my own work.

  \section*{Problem 2}

  \begin{enumerate}[label=(\alph*)]
  \item Show that $C$ could be derived within given knowledge base $\mathrm{KB}$.
  \begin{enumerate}[label=(\roman*)]
  \item Firstly, convert $\mathrm{KB}$ into CNF.
  \begin{align*}
    \frac{(A \lor B) \rightarrow C}{\lnot(A \lor B) \lor C} & \quad (\because P \rightarrow Q \text{ is equivalent to } \lnot P \lor Q)\\
    \frac{\lnot(A \lor B) \lor C}{(\lnot A \land \lnot B) \lor C} & \quad (\because \lnot(P \lor Q) \text{ is equivalent to } (\lnot P \land \lnot Q)) \\
    \frac{(\lnot A \land \lnot B) \lor C}{(\lnot A \lor C) \land (\lnot B \lor C)} & \quad (\because (P \land Q) \lor R \text{ is equivalent to } (P \lor R) \land (Q \lor R))
  \end{align*}
  Therefore, given knowledge base $\mathrm{KB}$ can be expressed in the following CNF.
  \[\mathrm{KB}' = \{(\lnot A \lor C) \land (\lnot B \lor C), A\}\]
  \item Lastly, inference from the derived formula.
  \begin{align*}
    \{(\lnot A \lor C) \land (\lnot B \lor C), A\} &\iff \{(A \rightarrow C) \land (B \rightarrow C), A\} \\
    &\iff \{A \rightarrow C, B \rightarrow C, A\}
  \end{align*}
  \end{enumerate}
  From $\mathrm{(ii)}$, $\mathrm{KB}'$ is equivalent to $\{A \rightarrow C, B \rightarrow C, A\}$, and it satisfies $\frac{A \rightarrow C, A}{C}$ (Modus ponens). It means $C$ can be derived from $\mathrm{KB}'$ by modus ponens. Also, using $\mathrm{(i)}$, $\mathrm{KB}'$ is equivalent to $\mathrm{KB}$, thus $C$ can be derived within $\mathrm{KB}$. \quad $\square$



  \end{enumerate}

\end{document}

